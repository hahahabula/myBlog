% Options for packages loaded elsewhere
\PassOptionsToPackage{unicode}{hyperref}
\PassOptionsToPackage{hyphens}{url}
\PassOptionsToPackage{dvipsnames,svgnames,x11names}{xcolor}
%
\documentclass[
  letterpaper,
  DIV=11,
  numbers=noendperiod]{scrartcl}

\usepackage{amsmath,amssymb}
\usepackage{iftex}
\ifPDFTeX
  \usepackage[T1]{fontenc}
  \usepackage[utf8]{inputenc}
  \usepackage{textcomp} % provide euro and other symbols
\else % if luatex or xetex
  \usepackage{unicode-math}
  \defaultfontfeatures{Scale=MatchLowercase}
  \defaultfontfeatures[\rmfamily]{Ligatures=TeX,Scale=1}
\fi
\usepackage{lmodern}
\ifPDFTeX\else  
    % xetex/luatex font selection
\fi
% Use upquote if available, for straight quotes in verbatim environments
\IfFileExists{upquote.sty}{\usepackage{upquote}}{}
\IfFileExists{microtype.sty}{% use microtype if available
  \usepackage[]{microtype}
  \UseMicrotypeSet[protrusion]{basicmath} % disable protrusion for tt fonts
}{}
\makeatletter
\@ifundefined{KOMAClassName}{% if non-KOMA class
  \IfFileExists{parskip.sty}{%
    \usepackage{parskip}
  }{% else
    \setlength{\parindent}{0pt}
    \setlength{\parskip}{6pt plus 2pt minus 1pt}}
}{% if KOMA class
  \KOMAoptions{parskip=half}}
\makeatother
\usepackage{xcolor}
\setlength{\emergencystretch}{3em} % prevent overfull lines
\setcounter{secnumdepth}{3}
% Make \paragraph and \subparagraph free-standing
\makeatletter
\ifx\paragraph\undefined\else
  \let\oldparagraph\paragraph
  \renewcommand{\paragraph}{
    \@ifstar
      \xxxParagraphStar
      \xxxParagraphNoStar
  }
  \newcommand{\xxxParagraphStar}[1]{\oldparagraph*{#1}\mbox{}}
  \newcommand{\xxxParagraphNoStar}[1]{\oldparagraph{#1}\mbox{}}
\fi
\ifx\subparagraph\undefined\else
  \let\oldsubparagraph\subparagraph
  \renewcommand{\subparagraph}{
    \@ifstar
      \xxxSubParagraphStar
      \xxxSubParagraphNoStar
  }
  \newcommand{\xxxSubParagraphStar}[1]{\oldsubparagraph*{#1}\mbox{}}
  \newcommand{\xxxSubParagraphNoStar}[1]{\oldsubparagraph{#1}\mbox{}}
\fi
\makeatother


\providecommand{\tightlist}{%
  \setlength{\itemsep}{0pt}\setlength{\parskip}{0pt}}\usepackage{longtable,booktabs,array}
\usepackage{calc} % for calculating minipage widths
% Correct order of tables after \paragraph or \subparagraph
\usepackage{etoolbox}
\makeatletter
\patchcmd\longtable{\par}{\if@noskipsec\mbox{}\fi\par}{}{}
\makeatother
% Allow footnotes in longtable head/foot
\IfFileExists{footnotehyper.sty}{\usepackage{footnotehyper}}{\usepackage{footnote}}
\makesavenoteenv{longtable}
\usepackage{graphicx}
\makeatletter
\newsavebox\pandoc@box
\newcommand*\pandocbounded[1]{% scales image to fit in text height/width
  \sbox\pandoc@box{#1}%
  \Gscale@div\@tempa{\textheight}{\dimexpr\ht\pandoc@box+\dp\pandoc@box\relax}%
  \Gscale@div\@tempb{\linewidth}{\wd\pandoc@box}%
  \ifdim\@tempb\p@<\@tempa\p@\let\@tempa\@tempb\fi% select the smaller of both
  \ifdim\@tempa\p@<\p@\scalebox{\@tempa}{\usebox\pandoc@box}%
  \else\usebox{\pandoc@box}%
  \fi%
}
% Set default figure placement to htbp
\def\fps@figure{htbp}
\makeatother

\usepackage{xeCJK}
\KOMAoption{captions}{tableheading}
\makeatletter
\@ifpackageloaded{tcolorbox}{}{\usepackage[skins,breakable]{tcolorbox}}
\@ifpackageloaded{fontawesome5}{}{\usepackage{fontawesome5}}
\definecolor{quarto-callout-color}{HTML}{909090}
\definecolor{quarto-callout-note-color}{HTML}{0758E5}
\definecolor{quarto-callout-important-color}{HTML}{CC1914}
\definecolor{quarto-callout-warning-color}{HTML}{EB9113}
\definecolor{quarto-callout-tip-color}{HTML}{00A047}
\definecolor{quarto-callout-caution-color}{HTML}{FC5300}
\definecolor{quarto-callout-color-frame}{HTML}{acacac}
\definecolor{quarto-callout-note-color-frame}{HTML}{4582ec}
\definecolor{quarto-callout-important-color-frame}{HTML}{d9534f}
\definecolor{quarto-callout-warning-color-frame}{HTML}{f0ad4e}
\definecolor{quarto-callout-tip-color-frame}{HTML}{02b875}
\definecolor{quarto-callout-caution-color-frame}{HTML}{fd7e14}
\makeatother
\makeatletter
\@ifpackageloaded{caption}{}{\usepackage{caption}}
\AtBeginDocument{%
\ifdefined\contentsname
  \renewcommand*\contentsname{Table of contents}
\else
  \newcommand\contentsname{Table of contents}
\fi
\ifdefined\listfigurename
  \renewcommand*\listfigurename{List of Figures}
\else
  \newcommand\listfigurename{List of Figures}
\fi
\ifdefined\listtablename
  \renewcommand*\listtablename{List of Tables}
\else
  \newcommand\listtablename{List of Tables}
\fi
\ifdefined\figurename
  \renewcommand*\figurename{Figure}
\else
  \newcommand\figurename{Figure}
\fi
\ifdefined\tablename
  \renewcommand*\tablename{Table}
\else
  \newcommand\tablename{Table}
\fi
}
\@ifpackageloaded{float}{}{\usepackage{float}}
\floatstyle{ruled}
\@ifundefined{c@chapter}{\newfloat{codelisting}{h}{lop}}{\newfloat{codelisting}{h}{lop}[chapter]}
\floatname{codelisting}{Listing}
\newcommand*\listoflistings{\listof{codelisting}{List of Listings}}
\makeatother
\makeatletter
\makeatother
\makeatletter
\@ifpackageloaded{caption}{}{\usepackage{caption}}
\@ifpackageloaded{subcaption}{}{\usepackage{subcaption}}
\makeatother
\makeatletter
\@ifpackageloaded{algorithm}{}{\usepackage{algorithm}}
\makeatother
\makeatletter
\@ifpackageloaded{algpseudocode}{}{\usepackage{algpseudocode}}
\makeatother
\makeatletter
\@ifpackageloaded{caption}{}{\usepackage{caption}}
\makeatother

\usepackage{bookmark}

\IfFileExists{xurl.sty}{\usepackage{xurl}}{} % add URL line breaks if available
\urlstyle{same} % disable monospaced font for URLs
\hypersetup{
  pdftitle={GBDT模型},
  pdfauthor={Hahabula},
  colorlinks=true,
  linkcolor={blue},
  filecolor={Maroon},
  citecolor={Blue},
  urlcolor={Blue},
  pdfcreator={LaTeX via pandoc}}


\title{GBDT模型}
\author{Hahabula}
\date{2025-02-07}

\begin{document}
\maketitle

\floatname{algorithm}{算法}

\renewcommand*\contentsname{目录}
{
\hypersetup{linkcolor=}
\setcounter{tocdepth}{3}
\tableofcontents
}

GBDT (Gradient Boosting Decision Tree),
梯度提升树。它是一种基于决策树的集成算法。

\begin{itemize}
\item
  基本结构:决策树组成的树林
\item
  学习方式:梯度提升
\end{itemize}

通过构造一组弱的学习(树),并把多颗决策树的结果累加起来作为最终的预测输出。该算法将决策树与集成思想进行了有效结合。

\section{GBDT详解}\label{gbdtux8be6ux89e3}

接下来将详细介绍GBDT模型。

\subsection{GBDT的原理}\label{gbdtux7684ux539fux7406}

\begin{itemize}
\tightlist
\item
  所有弱分类器的结构相加等于预测值
\item
  每次都以当前预测值为基准,下一个弱分类器去拟合误差函数对预测值的误差
\item
  GBDT 的弱分类器使用的是树模型
\end{itemize}

\subsection{前向分布算法}\label{sec-ux524dux5411ux5206ux5e03ux7b97ux6cd5}

许多加法模型都有如下形式

\[
f(x)=\sum_{i=1}^M\beta_m b(x,\gamma_m)
\]

\(\beta_m\) 是系数, \(b(x,\gamma_m)\) 是基函数并带有一组参数
\(\gamma_m\)。 这类模型的参数的求解大都采用极小化如下损失函数:

\[
\sum_{i=1}^nL(y_i,\sum_{m=1}^M\beta_mb(x,\gamma_m))
\]

但是当样本足够多,问题较为复杂时,如果直接求解,则要估计的参数将会有
\(M+\sum_{m=1}^M\dim(\gamma_m)\)
个,求解十分耗费算力,于是前向分布算法(Forward stagewise additive
modeling)被提出了。
前向分布算法将原来的问题转换为------一步一步的估计基函数项,当在估计某一基函数项
\(\beta_mb(x,\gamma_m)\) 时,其之前的
\(\beta_ib(x,\gamma_i),i=1,2,\ldots,m-1\)
将作为定值,不再被改变,故该算法的第m步即是取解决如下下问题:

\[
\begin{aligned}
\min& \sum_{i=1}^nL(y_i,f_m(x))\\
f_m(x)&=f_{m-1}(x)+\beta_mb(x,\gamma_m)
\end{aligned}
\]

其中 \(f_{m-1}(x)\) 已知。 当我们假定采用平方损失函数时,我们有:

\[
\begin{aligned}
Loss&=\sum_{i=1}^n\frac 12[y_i-(f_{m-1}(x)+\beta_mb(x,\gamma_m))]^2\\
&=\sum_{i=1}^n\frac 12[(y_i-f_{m-1}(x))-\beta_mb(x,\gamma_m)]^2\\
&=\sum_{i=1}^n\frac 12(r_{mi}-\beta_mb(x,\gamma_m))^2
\end{aligned}
\]

\(r_{mi}=y_i-f_{m-1}(x_i)\) 是在第m-1
步拟合后模型的残差,从损失函数得第m步的本质在于利用第m个基函数对上一步所得残差进行拟合,但这种理解是建立在损失函数为平方损失的情况下。

\begin{itemize}
\item
  前向的意义:``前向''指的是逐步进行的过程,即算法是从零开始,逐步向解的方向构建和优化。每一步添加一个新的基函数或模型参数,以逐渐逼近最终的目标。与之相对的概念是``后向''(Backward),后向算法通常从一个复杂的模型开始,然后逐步移除不必要的成分,而前向算法则是从简单的模型开始逐步加成。
\item
  分布的含义:``分布''或''阶段性''(Stagewise)意味着算法在每一步迭代时,只增加一个新的基函数或参数,而不会对之前的基函数或参数进行重新调整。也就是说,每次迭代仅调整新增的模型成分,已添加的部分在整个过程中保持不变。这种特性使得算法较为保守,每次变化较小,且有利于控制模型的复杂度和避免过拟合。
\end{itemize}

\section{参数估计}\label{ux53c2ux6570ux4f30ux8ba1}

假设有一个加法模型:

\[
F(\mathbf x;\{\beta_m,\mathbf a_m\}_1^M)=\sum_{i=1}^M\beta_mh(\mathbf x;\mathbf a_m)
\]

\(h(\mathbf x;\mathbf a_m)\)
是一个参数比较简单的基函数;\(\mathbf a=\{a_1,a_2,\ldots\}\)

\subsection{数值优化}\label{ux6570ux503cux4f18ux5316}

需要估计的参数为:

\[
\mathbf P^*=\arg\min_{\mathbf P}\Phi(\mathbf P)
\]

\[
\Phi(\mathbf P)=E_{y,\mathbf x}L(y,F(\mathbf x;\mathbf P))
\]

最优的模型为:

\[
F^*(\mathbf x)=F(\mathbf x;\mathbf P^*)
\]

参数的组成:

\[
\mathbf P^*=\sum_{m=0}^M\mathbf p_m
\]

对于参数构成的理解:即所有参数的列向量,每一次拟合模型都相当于往
\(\mathbf P\) 的某些列上加值/更新值。

\subsection{最速下降法(Steepest-descent)}\label{ux6700ux901fux4e0bux964dux6cd5steepest-descent}

由前面讨论可得,有如下优化问题:

\[
\min_{\mathbf P}\Phi(\mathbf P)
\]

则由梯度下降法得到的梯度为:

\[
\mathbf g_m=\nabla\Phi(\mathbf P)|_{\mathbf P=\mathbf P_{m-1}}=[\frac{\partial\Phi(\mathbf P)}{\partial P_j}|_{\mathbf P =\mathbf P_{m-1} }]^T
\]

由于 \(\mathbf P_{m-1}=\sum_{i=0}^{m-1}\mathbf p_i\),
则得到如下迭代公式:

\[
\mathbf P_m=\mathbf P_{m-1}+\mathbf p_m
\]

而由最速下降法得:

\[
\mathbf p_m=-\rho_m\mathbf g_m
\]

由一维线搜索得步长 \(\rho_m\) 有下式决定:
由前面讨论可得,有如下优化问题:

\[
\min_{\mathbf P}\Phi(\mathbf P)
\]

则由梯度下降法得到的梯度为:

\[
\mathbf g_m=\nabla\Phi(\mathbf P)|_{\mathbf P=\mathbf P_{m-1}}=[\frac{\partial\Phi(\mathbf P)}{\partial P_j}|_{\mathbf P =\mathbf P_{m-1} }]^T
\]

由于 \(\mathbf P_{m-1}=\sum_{i=0}^{m-1}\mathbf p_i\),
则得到如下迭代公式:

\[
\mathbf P_m=\mathbf P_{m-1}+\mathbf p_m
\]

而由最速下降法得:

\[
\mathbf p_m=-\rho_m\mathbf g_m
\]

由一维线搜索得步长 \(\rho_m\) 有下式决定:

\[
\rho_m=\arg\min_\rho\Phi(\mathbf P_{m-1}-\rho\mathbf g_m)
\]

\subsection{函数空间上的数值优化}\label{ux51fdux6570ux7a7aux95f4ux4e0aux7684ux6570ux503cux4f18ux5316}

假定某函数由多个基函数相加而成,即加法模型:

\[
F^*(x)=\sum_{m=0}^Mf_m(x)
\]

有如下优化问题:

\[
\min\Phi(F(x))=E_y[L(y,F(x))|x]
\]

\(F_m(x)\) 是迭代的类似于 \(x_k=x_{k-1}+\alpha_kd_k\) 则由最速下降法得:

\[
f_m(\mathbf x)=-\rho_m\mathbf g_m(\mathbf x)
\]

\[
\mathbf g_m(\mathbf x)=E_y[\frac{\partial L(y,F(\mathbf x))}{\partial F(\mathbf x)}|\mathbf x]_{\mathbf F(x)=\mathbf F_{m-1}(\mathbf x)}
\]

步长 \(\rho_m\) 由下列精确一维线搜索决定:

\[
\rho_m = \arg\min_\rho E_{y,\mathbf x}L(y,F_{m-1}(\mathbf x)-\rho\mathbf g_m(\mathbf x))
\]

\subsection{有限维数据------梯度提升算法}\label{ux6709ux9650ux7ef4ux6570ux636eux68afux5ea6ux63d0ux5347ux7b97ux6cd5}

从参数的角度而言,已知样本数据
\(\{y_i,\mathbf x_i\}\),模型为加法模型,则有如下参数优化问题:

\begin{equation}\phantomsection\label{eq-ux53c2ux6570ux4f18ux5316ux9650ux5236ux6761ux4ef6}{
\{\beta_m^*,\mathbf a_m^*\}_1^M=\arg\min_{\{\beta_m,\mathbf a_m\}_1^M}\sum_{i=1}^NL(y_i,\sum_{m=1}^M\beta_mh(\mathbf x_i;\mathbf a_m))
}\end{equation}

要直接求解上述参数优化问题将会比较棘手,因为需要在一个优化问题中求解
\(M\cdot(1+\dim (\mathbf a_m))\) 个参数,参数个数庞大,我们考虑使用
Section~\ref{sec-前向分布算法}
中的前向分布算法,其将一个参数优化问题转换为一个函数优化问题。
利用前向分布算法后,此时需要解决函数优化问题,即对该优化问题的每一步有:

\[
F^*=\arg\min_{F}\sum_{i=1}^NL(y_i,F(\mathbf x_i))
\]

\[
F_m(\mathbf x)=F_{m-1}(\mathbf x)+\beta_m h(\mathbf x;\mathbf a_m)
\]

由最速下降法得:

\[
h(\mathbf x_i;\mathbf a_m)=-\mathbf g_m(\mathbf x_i)=-[\frac{\partial L(y_i,F(\mathbf x_i))}{\partial F(\mathbf x_i)}]_{F(\mathbf x)=F_{m-1}(\mathbf x)}
\]

但是 \(h(\mathbf x;\mathbf a_m)\) 是一个在 \(\mathcal D_{\mathbf x}\)
上均有定义的函数,而 \(\mathbf g_m(\mathbf x_i)\) 仅在
\(\{\mathbf x_1,\ldots,x_N\}\) 处有定义,故上式并不能直接将
\(h(\mathbf x;\mathbf a_m)\) 直接确定下来,需要采用其他方式让
\(h(\mathbf x;\mathbf a_m)\) 去贴合负梯度方向。 一种想法是在
\(h(\mathbf x;\mathbf a_m)\) 函数族中找到一个
\(h^*(\mathbf x;\mathbf a_m)\) 使得
\(\mathcal h_m=\{h(\mathcal x_i;\mathcal a_m)\}_1^N\) 与
\(-\mathcal g_m\) 最平行,则有如下优化问题:

\[
\mathbf a_m=\arg\min_a\sum_{i=1}^N[-\mathbf g_m(\mathbf x_i)-\alpha h(\mathbf x_i;\mathbf a)]^2
\]

\begin{tcolorbox}[enhanced jigsaw, arc=.35mm, left=2mm, title=\textcolor{quarto-callout-tip-color}{\faLightbulb}\hspace{0.5em}{寻找负梯度方向说明}, coltitle=black, opacitybacktitle=0.6, colframe=quarto-callout-tip-color-frame, toptitle=1mm, bottomtitle=1mm, opacityback=0, leftrule=.75mm, colbacktitle=quarto-callout-tip-color!10!white, titlerule=0mm, colback=white, bottomrule=.15mm, toprule=.15mm, rightrule=.15mm, breakable]

\begin{itemize}
\item
  此处的思想是回归的思想,但是没有截距项因为我们要的就是尽量平行。两向量平行的数学定义是
  \(-\mathbf g_m=\alpha \mathbf h_m\)
\item
  \(\alpha\) 也是该优化问题所要求解的
\end{itemize}

\end{tcolorbox}

一旦确定了 \(h(\mathbf x;\mathbf a_m)\)
后,我们便可利用精确一维线搜索去求解步长 \(\beta_m\):

\[
\beta_m=\arg\min_{\beta}\sum_{i=1}^NL(y_i,F_{m-1}(\mathbf x_i+\beta h(\mathbf x_i;\mathbf a_m)))
\]

确定了 \(\beta_m,\mathbf a_m\) 后便可得到:

\[
F_m(\mathbf x)=F_{m-1}(\mathbf x)+\beta_m h(\mathbf x;\mathbf a_m)
\]

在现实数据中并不是直接去找 Equation~\ref{eq-参数优化限制条件} 下的
\(h(\mathbf x;\mathbf a_m)\) ,而是将 \(h(\mathbf x;\mathbf a_m)\)
去拟合伪响应 \(\{\tilde{y}_i=-\mathbf g_m(\mathbf x_i)\}_i^N\),
这使得函数优化问题转化为了最小二乘函数优化问题。

\begin{algorithm}[H]
\caption{梯度提升算法}
\begin{algorithmic}[1]
\State 初始化模型 $F_0(\mathbf{x}) = \arg\min_{\rho} \sum_{i=1}^N L(y_i, \rho)$
\For{$m = 1$ 到 $M$}
    \State 计算伪残差:$\tilde y_i = -\left[\frac{\partial L(y_i, F(\mathbf{x}_i))}{\partial F(\mathbf{x}_i)}\right]_{F=F_{m-1}}$
    \State 拟合一个基学习器 $h_m(\mathbf{x};\mathbf a_m)$ 来拟合伪残差 $\tilde y_i:\mathbf a_m=\arg\min_{\mathbf a}\sum_{i=1}^N[-\mathbf g_m(\mathbf x_i)-\alpha h(\mathbf x_i;\mathbf a)]^2$
    \State 计算最佳步长:$\beta_m=\arg\min_{\beta}\sum_{i=1}^NL(y_i,F_{m-1}(\mathbf x_i+\beta h(\mathbf x_i;\mathbf a_m)))$
    \State 更新模型:$F_m(\mathbf{x}) = F_{m-1}(\mathbf{x}) + \beta_m h_m(\mathbf{x};\mathbf a_m)$
\EndFor
\State \Return 最终模型 $F_M(\mathbf{x})$
\end{algorithmic}
\end{algorithm}

\section{GBDT算法}\label{gbdtux7b97ux6cd5}

GBDT算法如下:

\begin{itemize}
\tightlist
\item
  要拟合的模型是:
  \(f(\mathbf{x})=f_0(\mathbf{x})+\sum_{m=1}^M\beta_m\sum_{j=1}^J\Upsilon_{jm}I(\mathbf{x}\in R_{im}),\mathbf{x}=(x_1,x_2,\ldots,x_p)^T\)
\item
  数据结构:即有n个样本,p个特征
  \(X=\begin{pmatrix}\mathbf{x}_1\\\vdots\\\mathbf{x}_n\end{pmatrix}\)
\end{itemize}

\begin{enumerate}
\def\labelenumi{\arabic{enumi}.}
\item
  初始化学习器 该弱学习器不具有任何参数即 \(f_0(\mathbf{x}_i)=c\) 则
  \(f_0(\mathbf x)=\arg\min_c\sum_{i=1}^n L(y_i,c)\)
  ,一般情况下可以取平方损失即 \(L(y_i,c)=\frac 12(y_i-c)^2\) 求导可得
  \(\frac{\partial \sum_{i=1}^N(\frac 12 (y_i-c)^2)}{\partial c}=-\sum_{i=1}^N(y_i-c)=0\Rightarrow c=\frac 1N\sum_{i=1}^N y_i\),
  即在损失为平方损失的情况下,初始化的弱学习器将会使得各样本的初始值为它们的均值。
\item
  构建后续学习器 对 \(m=1,2,\ldots,M\)
  ,即构建M个决策树,假设取平方损失,则对于第m步而言损失函数有如下形式:
  \(Loss=\sum_{i=1}^n\frac 12(y_i-f(\mathbf{x}_i))^2=L(\mathbf{y},f(\mathbf{x}))|_{f(\mathbf{x})=f_{m}(\mathbf{x})}=\frac12||\mathbf y-F_m(X)||_2^2\)
  \(F_m(X)=\begin{pmatrix}f_m(\mathbf x_1)\\\vdots\\f_m(\mathbf x_n)\end{pmatrix},B(\mathbf x,\gamma_m)=\begin{pmatrix}b(\mathbf x_1,\gamma_m)\\\vdots\\b(\mathbf x_n,\gamma_m)\end{pmatrix},\mathbf y=\begin{pmatrix}y_1\\\vdots\\ y_n\end{pmatrix}\)

  第m步的模型为:
  \(f_m(\mathbf{x})=f_{m-1}(\mathbf{x})+\sum_{j=1}^J\Upsilon_{jm}I(\mathbf{x}\in R_{jm})=f_{m-1}(\mathbf{x})+b(\mathbf{x},\gamma_m)\)

  由最速下降法得第m步的迭代公式为(以 \(F_m(X)\)
  为整体):\(F_m(X)=F_{m-1}(X)-\alpha_{m-1}g_{m-1},g_{m-1}=\nabla_{F_{m-1}(X)} L(\mathbf{y},F(X))\)
  令 \(\alpha_k=\mathbf 1_n,k=1,2,\ldots,m-1\), 则有当
  \(F_m(X)=F_{m-1}(X)-\frac{\partial L(\mathbf y,F(X))}{\partial F(X)}|_{F(X)=F_{m-1}(X)}\)

  而由 \(f_m(\mathbf x)\) 的形式可得
  \(B(\mathbf x,\gamma_m)=-\frac{\partial L(\mathbf y,F(X))}{\partial F(X)}|_{F(X)=F_{m-1}(X)}\)
  由Matrix
  cookbook书得有:\(\frac{\partial \mathbf x^T\mathbf a}{\partial \mathbf x}=\frac{\partial \mathbf a^T\mathbf x}{\partial \mathbf x}=\mathbf a\)
  和
  \(\frac{\partial \mathbf b^TX^TX\mathbf c}{\partial X}=X(\mathbf b\mathbf c^T+\mathbf c\mathbf b^T)\)
  \(-\frac{\partial L(\mathbf y,F(X))}{\partial F(X)}|_{F(X)=F_{m-1}(X)}=-\frac 12[-2\mathbf y+2F_{m-1}(X)]=\mathbf y-F_{m-1}(X)\)

  也即
  \(b(\mathbf x_i,\gamma_m)=r_{im}=y_i-f_{m-1}(\mathbf x)\),即第m颗决策树需要拟合的是上一颗决策树构建后所得的残差。
\item
  用 \((\mathbf x_i,r_{im})_{i=1}^n\)
  建立第m颗树后,需要利用一维线搜索解决如下一维优化问题以计算
  \(\beta_m\):
  \(\beta_m=\arg\min_{\beta}\sum_{i=1}^NL(y_i,F_{m-1}(\mathbf x_i+\beta h(\mathbf x_i;\mathbf a_m)))\)
\end{enumerate}

\texttt{最终的学习器}即为:\(f(\mathbf x)=f_0(\mathbf x)+\sum_{m=1}^M\sum_{j=1}^J\Upsilon_{jm}I(\mathbf x\in R_{jm})\)




\end{document}
